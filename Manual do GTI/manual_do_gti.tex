\documentclass{article}
\usepackage[utf8]{inputenc}
\usepackage[portuguese]{babel}

\title{Manual do GTI de Escrita Científica
	\author{
		Lucas Aquino de Assis
		\and
		Ana Carla Quallio Rosa
		\and
		INSERIR NOME GUILHERME
	}
	\thanks{Grupo de Trabalho criado pelo Laboratório de Aprendizado de Máquina e Imagens Aplicados à Indústria (LAMIA)}
}

\begin{document}

\begin{titlepage}
\maketitle
\end{titlepage}

\tableofcontents

\section{Objetivos}
\subsection{Objetivo principal}
Este Grupo de Trabalho Interno (GTI) tem como objetivo principal apoiar a escrita acadêmica e científica do LAMIA, por meio da criação e adoção de uma metodologia própria de escrita. Ou seja, este grupo irá construir e estabelecer para o laboratório um conjunto de práticas, as quais irão auxiliar os integrantes a escreverem seus trabalhos, padronizando estruturas e indicando possíveis abordagens. 

\subsection{Funções específicas}
\begin{itemize}
  \item Auxiliar orientadores e alunos a documentarem a evolução de seus projetos para a futura escrita de artigos;
  \item Fundamentar uma metodologia científica que padronize a estrutura, organização e o fluxo de informações com base em projetos já feitos pelos alunos do laboratório;
  \item Monitorar periódicos e demais eventos que são pertinentes às temáticas de cada trabalho, alertando, assim, datas importantes, como data de envio, apresentação, entre outros.
\end{itemize}

\section{Ferramentas Utilizadas}
\subsection{LaTeX}
A ferramenta LaTeX é uma das mais importantes para o GTI, já que é a responsável por estruturar todos os documentos que serão produzidos. Com o LaTeX é possível focar no conteúdo dos documentos, e deixar o computador tomar conta da formatação, acelerando o processo de escrita, portanto é essencial que todos os membros do GTI aprendam a utilizar o LaTeX.

\subsection{Git}
O Git é um sistema de controle de versão que é essencial para o trabalho em conjunto dos integrantes do GTI, pois permite que vários integrantes trabalhem em diversos projetos ao mesmo tempo no repositório do GTI. É importante que todos os integrantes familiarizem-se com a ferramenta.

\subsection{Google Drive}
Todos os projetos e arquivos do GTI serão armazenados no Google Drive, e a organização das pastas é de responsabilidade de todos os membros.

\subsection{Google Planilhas}
Documentos importantes, como o Plano do GTI, estão em planilhas do Google Planilhas pois assim ficam organizados em várias abas.

\section{Regras de Convivência}

\end{document}